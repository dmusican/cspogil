\model{Invoke and Return}

Each statement in this program \emph{invokes} (or calls) a function.
At the end of a function, C++ \emph{returns} to where it was invoked.
The list of events on the right illustrates how the program runs.

\vspace{3ex}

\begin{minipage}{0.68\textwidth}
\begin{javabox}
void setup() {
    Serial.begin(9600);
}

void newLine() {
    Serial.println();
}

void threeLine() {
    newLine();
    newLine();
    newLine();
}

void loop() {
    Serial.println("First line.");
    threeLine();
    Serial.println("Second line.");

    while(true) {
    }
}
\end{javabox}
\end{minipage}
\hfill
\begin{minipage}{0.34\textwidth}
\begin{verbatim}

INVOKE println
RETURN to line 16
INVOKE threeLine
    INVOKE newLine
        INVOKE println
        RETURN to line 6
    RETURN to line 10
    INVOKE newLine
        INVOKE println
        RETURN to line 6
    RETURN to line 11
    INVOKE newLine
        INVOKE println
        RETURN to line 6
    RETURN to line 12
RETURN to line 17
INVOKE println
RETURN to line 18

\end{verbatim}
\end{minipage}



\Q \label{lines}
How many lines of source code invoke the \java{println} method? \ans{Three (lines 4, 6, 10)}
\vspace{1em}


\Q \label{times}
How many times is \java{println} invoked when the program runs? \ans{Five times}
\vspace{1em}

\Q For each \texttt{INVOKE} on the right, draw an arrow to the corresponding line of code.
(Plan ahead so that crossing lines will still be legible.) \ans{There should be 9 lines total.}
\vspace{1em}


\Q What is the output of the program? Please write \java{\\n} to show each newline character.

\begin{answer}[8em]
\vspace{-1ex}
\begin{javaans}
First line.\n
\n
\n
\n
Second line.\n
\end{javaans}
\end{answer}

\Q \label{times}
What purpose does the \java{while} loop serve at the end of the function
\java{loop}?
\vspace{10em}




\Q When C++ sees a name like \java{x}, \java{count}, or \java{newLine}, how can it tell whether it's a variable or a method? (Hint: syntax)

\begin{answer}[5em]
Methods have parentheses, and variables do not.
They are similar to functions in math  : when you see $g(x)$, you know that $g$ is a function and $x$ is a variable.
\end{answer}


\Q What is the difference between a function and a variable? What do they have in common?

\begin{answer}[5em]
All computer programs, regardless of language, consist of \emph{code} and \emph{data}.
Methods contain code (statements or instructions), whereas variables contain data (references or values).
\end{answer}


\Q In your own words, describe what functions are for. Why not just write everything in \java{main}?

\begin{answer}[5em]
Methods help organize the code into separate parts.
They also make it possible to write code once and use it multiple times.
\end{answer}

%%% Local Variables:
%%% mode: latex
%%% TeX-master: t
%%% End:
