\model{Arrays and Loops}

The real power of arrays is the ability to process them using loops, i.e., performing the same task for multiple elements.

\begin{javalst}
    for (int i = 0; i < length; i++) {
       // ... process array[i] ...
    }
\end{javalst}

Here is one specific example:

\begin{javalst}
  // Below are five pins that LEDS are connected to
  int NUM_LEDS = 5;
  int ledPins[] = {5,6,8,11,12}

  for (int i=0; i < NUM_LEDS; i++) {
    digitalWrite(ledPins[i], HIGH);
  }
\end{javalst}




\newpage

\Q \label{pairwiseMax} Finish the following program which should take an array
of pins representing LEDs, that repeatedly turns them all off, and then turns
them on one at a time (with a second in between each) so that at the end, all of
the LEDs are lit. Make sure to use \texttt{for} loops to do things repeatedly;
don't write the (almost) same line of code five times.


\begin{javalst}
const int NUM_LEDS = 5;
int leds[] = {3,5,6,9,12};

void setup()
{

  








}

void loop() {



  










  

  

}
\end{javalst}


\newpage


\Q Finish the following program which should play a melody with notes of
different frequencies and durations. Make sure to use \texttt{for} loops to do
things repeatedly; don't write the (almost) same line of code five times.


\begin{javalst}
const int BEEPER_PIN = 2;

// notes in the melody:
int melody[] = {262, 196, 196, 220, 196, 263, 220, 263};

// note durations: 4 = quarter note, 8 = eighth note, etc.:
int noteDurations[] = {4, 8, 8, 4, 4, 4, 4, 4 };

void setup() {



  
    
}

void loop() {





  

















}
\end{javalst}


%%% Local Variables:
%%% mode: latex
%%% TeX-master: "Act07-Arrays"
%%% End:
